\documentclass[12pt]{article}
\usepackage{times}
\usepackage{geometry}                % See geometry.pdf to learn the layout options. There are lots.
\geometry{letterpaper}                   % ... or a4paper or a5paper or ...
%\geometry{landscape}                % Activate for for rotated page geometry
%\usepackage[parfill]{parskip}    % Activate to begin paragraphs with an empty line rather than an indent
\usepackage{graphicx}
\usepackage{amssymb}
\usepackage{amsmath}
\usepackage{epstopdf}
\usepackage{wrapfig}
\usepackage[compress]{natbib}
%\usepackage[demo]{graphicx}
\usepackage{caption}
\usepackage{subcaption}
%\usepackage[square,comma,numbers,sort]{natbib}
% \bibpunct{(}{)}{;}{a}{}{,} % to follow the A&A style
\usepackage[pdftex, plainpages=false, colorlinks=true, linkcolor=blue, citecolor=blue, bookmarks=false]{hyperref}
\usepackage{setspace}
\usepackage{multicol}
\usepackage{sectsty}
\usepackage{url}
\usepackage{lipsum}
\usepackage[tiny,compact]{titlesec}
\usepackage{fancyhdr}
%\usepackage{deluxetable}
%OG
%\usepackage[font=footnotesize,labelfont=bf]{caption}
%NEW
\usepackage[font=normalsize,labelfont=bf]{caption}


\usepackage{verbatim}
\usepackage[super]{nth}
\usepackage{enumitem}
\usepackage{bbding}

\setlength{\textwidth}{6.5in}
\setlength{\oddsidemargin}{0.0cm}
\setlength{\evensidemargin}{0.0cm}
\setlength{\topmargin}{-0.5in}
\setlength{\headheight}{0.2in}
\setlength{\headsep}{0.2in}
\setlength{\textheight}{9.in}
%\setlength{\footskip}{-0.2in}
%\setlength{\voffset}{0.0in}
%\setlength{\tabcolsep}{1pt}

\newcommand{\todo}[1]{{\color{red}$\blacksquare$~\textsf{[TODO: #1]}}}

% The Title of this Whole Thing, kind of short title
\newcommand{\doctitle}{Neutrinos Transport Fidelity in Collapsars}

\sectionfont{\normalsize}
\subsectionfont{\normalsize}
\subsubsectionfont{\normalsize}
\singlespacing

\pagestyle{fancy}
\fancyhf{}
%\renewcommand{\headrulewidth}{0pt}

%header info
\lhead{\fancyplain{}{\doctitle}}
\rhead{\fancyplain{}{Brandon Barker, Jonah Miller}}
\rfoot{\fancyplain{}{\thepage}}

% \bibliographystyle{aasjournal}
% \bibliographystyle{nsf}
%\bibliographystyle{abbrv}
%\bibliographystyle{physrev}

%removes bib label%
%\makeatletter
%\renewcommand\@biblabel[1]{}
%\makeatother

\input macros.tex
\newcommand\aj{AJ}% 
          % Astronomical Journal 
\newcommand\araa{ARA\&A}% 
          % Annual Review of Astron and Astrophys 
\newcommand\apj{ApJ}% 
          % Astrophysical Journal 
\newcommand\apjl{ApJ}% 
          % Astrophysical Journal, Letters 
\newcommand\apjs{ApJS}% 
          % Astrophysical Journal, Supplement 
\newcommand\ao{Appl.~Opt.}% 
          % Applied Optics 
\newcommand\apss{Ap\&SS}% 
          % Astrophysics and Space Science 
\newcommand\aap{A\&A}% 
          % Astronomy and Astrophysics 
\newcommand\aapr{A\&A~Rev.}% 
          % Astronomy and Astrophysics Reviews 
\newcommand\aaps{A\&AS}% 
          % Astronomy and Astrophysics, Supplement 
\newcommand\azh{AZh}% 
          % Astronomicheskii Zhurnal 
\newcommand\baas{BAAS}% 
          % Bulletin of the AAS 
\newcommand\jrasc{JRASC}% 
          % Journal of the RAS of Canada 
\newcommand\memras{MmRAS}% 
          % Memoirs of the RAS 
\newcommand\mnras{MNRAS}% 
          % Monthly Notices of the RAS 
\newcommand\pra{Phys.~Rev.~A}% 
          % Physical Review A: General Physics 
\newcommand\prb{Phys.~Rev.~B}% 
          % Physical Review B: Solid State 
\newcommand\prc{Phys.~Rev.~C}% 
          % Physical Review C 
\newcommand\prd{Phys.~Rev.~D}% 
          % Physical Review D 
\newcommand\pre{Phys.~Rev.~E}% 
          % Physical Review E 
\newcommand\prl{Phys.~Rev.~Lett.}% 
          % Physical Review Letters 
\newcommand\pasp{PASP}% 
          % Publications of the ASP 
\newcommand\pasj{PASJ}% 
          % Publications of the ASJ 
\newcommand\qjras{QJRAS}% 
          % Quarterly Journal of the RAS 
\newcommand\skytel{S\&T}% 
          % Sky and Telescope 
\newcommand\solphys{Sol.~Phys.}% 
          % Solar Physics 
\newcommand\sovast{Soviet~Ast.}% 
          % Soviet Astronomy 
\newcommand\ssr{Space~Sci.~Rev.}% 
          % Space Science Reviews 
\newcommand\zap{ZAp}% 
          % Zeitschrift fuer Astrophysik 
\newcommand\nat{Nature}% 
          % Nature 
\newcommand\iaucirc{IAU~Circ.}% 
          % IAU Cirulars 
\newcommand\aplett{Astrophys.~Lett.}% 
          % Astrophysics Letters 
\newcommand\apspr{Astrophys.~Space~Phys.~Res.}% 
          % Astrophysics Space Physics Research 
\newcommand\bain{Bull.~Astron.~Inst.~Netherlands}% 
          % Bulletin Astronomical Institute of the Netherlands 
\newcommand\fcp{Fund.~Cosmic~Phys.}% 
          % Fundamental Cosmic Physics 
\newcommand\gca{Geochim.~Cosmochim.~Acta}% 
          % Geochimica Cosmochimica Acta 
\newcommand\grl{Geophys.~Res.~Lett.}% 
          % Geophysics Research Letters 
\newcommand\jcp{J.~Chem.~Phys.}% 
          % Journal of Chemical Physics 
\newcommand\jgr{J.~Geophys.~Res.}% 
          % Journal of Geophysics Research 
\newcommand\jqsrt{J.~Quant.~Spec.~Radiat.~Transf.}% 
          % Journal of Quantitiative Spectroscopy and Radiative Trasfer 
\newcommand\memsai{Mem.~Soc.~Astron.~Italiana}% 
          % Mem. Societa Astronomica Italiana 
\newcommand\nphysa{Nucl.~Phys.~A}% 
          % Nuclear Physics A 
\newcommand\physrep{Phys.~Rep.}% 
          % Physics Reports 
\newcommand\physscr{Phys.~Scr}% 
          % Physica Scripta 
\newcommand\planss{Planet.~Space~Sci.}% 
          % Planetary Space Science 
\newcommand\procspie{Proc.~SPIE}% 
          % Proceedings of the SPIE 
\newcommand\jcphys{J.~Comp.~Phys.}% 
          % Journal of Computational Physics

\let\astap=\aap 
\let\apjlett=\apjl 
\let\apjsupp=\apjs 
\let\applopt=\ao 


%\titlespacing*{\section}{0in}{0.2in}{0in}
%\titlespacing*{\subsection}{0in}{0.1in}{0in}
\titleformat*{\subsection}{\itshape}
%\titlespacing*{\subsubsection}{0in}{0.in}{0in}
\titleformat*{\subsubsection}{\itshape}
\setlength{\abovecaptionskip}{3pt}

\begin{document}

\setcounter{page}{1} \pagenumbering{arabic} \renewcommand{\thepage}
           {\arabic{page} }%\pagestyle{plain}

\begin{center}
{\bf Impacts of neutrino transport fiedlity on collapsar outcomes} \vspace{-0.15in}
%\section*{Project Narrative}
\end{center}

%\begin{center}
%    Michael A. Pajkos\\
%\end{center}

% \section{Preparation and Past Research}

% \subsection{Overview}

% is this intro section too long..? not enough details??
Discerning the origins of the elements (nucleosynthesis) has been an active area of investigation dating back to the earliest civilizations.
We now know that the primary elemental production sites are astrophysical in nature.
Most of the naturally occuring elements heavier than Helium are known to originate from the deaths of stars \citep{burbidge:1957}.
This accounts for most of the elements up to iron.
Above this, elements are produced by more exoxtic means.
Chief among them is the so-called rapid neutron capture process (r-process), whose's astrophysical site of origin is still an area of active interdisciplinary study \citep{horowitz:2019}.
It has recently been shown that the r-process occurs in the mergers of two neutron stars \citep{abbott:2017a}, but evidence suggests that another site may be needed \citep{martinez-pinedo:2014}.
One proposed site is a collapsar \citep{siegel:2019}-- the product of the collapse of a massive star producing a black hole in its core \citep{Woosley1993, MacFadyen_1999}.
These systems, which include a black hole inside of a massive star accreting onto it, are not so unlike the conditions present in a neutron star merger and may produce conditions amenable to forming r-process elements.
%JMM: This sentence feels a little awkward and I ran out of space, so I cut it.
%It has been shown that, under certain assumptions, r-process elements can be formed in collapsars.

% how to say this? better to use fermions than electrons?
Critical ingredients to any nucleosynthesis calculation are the fractions of available electrons and hadrons in the matter.
These are governed, in part, by neutrino phsyics whose interactions modify the available hadrons in the matter.
It is common in studies of collapsar nucleosynthesis to minimalize neutrino transport to a so-called leakage scheme \citep{siegel:2019},
% JMM: This is a little confusing as the lightbulb model and ray-by-ray are
% things and different than each other and leakage.  The way leakage
% works is that the energy and Ye in each zone have source terms from
% emission and absorption, but there is no radiation field to set
% them or be set by them. Rather, effective emission and absorption
% are approximately computed by comparing the free-streaming or equilibrium
% rates to an estimated optical depth, which is often, but not necessarily,
% a line integral through the simulation, neglecting multi-D or GR effects.
% but this trivializes the transport to a simple ``lightbulb'' and critical effects in setting the radiation field are missed.
where emission and absorption are treated locally by comparing reaction rates to approximate optical depths.
However this misses any effects radiation transport may have on the matter field, such as gravitational redshift, scattering, and multidimensional effects.
Recently, it has been shown that higher fidelity neutrino physics may alter the nucleosynthesis and dynamics in the collapsar \citep{miller:2020}.
However, the extent to which how and where high fidelity neutrino transport impacts the outcomes of a collapsar is still relatively uncertain.
A systematic landscape study, varying both collapsar parameters and neutrino transport fidelity, is needed to fully understand the impacts of neutirno transport fidelity on collapsar outcomes.

% JMM: One thing you don't talk about here is that stellar parameters
% matter quite a lot, and possibly swamp everything else. A big
% challenge with Pheobus will be to vary stellar structure and initial
% conditions. Rather than starting with a disk, we'll want to see if
% we can make one form dynamically.

Using the new 3D general relativistic radiation magnetohydrodynamic (GRRMHD) code Pheobus currently in development at Los Alamos National Laboratory (LANL), I will  perform a suite of collapsar simulations with varying neutrino transport methods and physical parameters.
% JMM: Phoebus doesn't currently contain a leakage scheme, though it wouldn't be hard to implement one.
Pheobus currently includes several transport schemes, including a moment method, a sophisticated Monte Carlo scheme \citep{nubhlight}, and a new hybrid method capable of spanning the range of possible optical depths \citep{MOCMC}.
This code is built upon the performance portable adaptive mesh refinement (AMR) software Parthenon \citep{grete:2022}, developed at LANL, supports GPUs, and is highly performant.
% JMM: I think a big science question here is also what happens when we use realistic initial conditions with GRRMHD. No one has done that yet, and we plan to.
By running simulations with realistic initial conditions and varying the neutrino transport algorithms, these simulations will
reveal the impact of neutrino transport and stellar structure on collapsar outcomes, focusing on nucleosynthesis and accretion disk dynamics.
I will also use these simulations to study the prospects for collapsar ejecta breaking out from the stellar envelope that entombs it.

The expertise at LANL, both astrophysical and computational, makes it the ideal setting to perform this study. 
During my visit, I will work with the Phoebus/Parthenon developers such as Jonah Miller, Josh Dolence, and Ben Ryan. I will also interact with the experts in supernovae and nucleosynthesis such as Chris Fryer and Matt Mumpower.
Three-dimensional simulations are generally prohibitively computationally expensive, restricting studies such as this.
%Phoebus, being built on the highly performant AMR library Parthenon, will allow us to push the limit of computational expensive down to new levels. % this sounds weird and cheesey 
The performance portability of Phoebus and Parthenon will allow us to leverage the mostly untapped GPU-based HPC systems now coming online and perform 3D collapsar models at unprecedented scale.
All of this, in turn, will enable us to make predictions more faithful to the underlying physics while discerning what affects are most important in collapsar settings.

% Throughout their lifetimes, stars create heavier and heavier elements in their cores, powering the spectacular views that we see in the night sky.
% Once these stars exhaust their fuel at the ends of their lives, the heavy elements that they created are returned to the interstellar medium to enrich the next generation of stars.
% One such stellar death is a core-collapse supernova, which is the explosion occuring when stars much more massive than the sun are no longer able to undergo nuclear fusion in their cores \citep{mezzacappa:2001, mezzacappa:2005, janka:2012a, janka:2016, burrows:2013, hix:2014, muller:2016, couch:2017}.
% These explosions, occuring for a still unknown fraction of very massive stars, are the primary sources of most of the elements between Helium and Iron.
% Some of these massive stars, however, will not form an explosion and instead form a black hole in their cores.
% These systems of a black hole and accreting stellar material are known as collapsars and thought to be a potential source of the rapid neutron capure process (r-process) which is responsible for creating the heaviest elements.

\newpage

\setcounter{page}{1} \pagenumbering{arabic} \renewcommand{\thepage}
           {Bibliography -- \arabic{page}}

\renewcommand\bibsection{\section*{References}}
\setlength{\bibsep}{2pt}

%\bibliographystyle{abbrv}
\bibliographystyle{unsrt}
\bibliography{ProjectNarrative}


\end{document}
